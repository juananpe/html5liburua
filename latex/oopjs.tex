\chapter{Objektuetara zuzendutako programazioa JavaScript-en}

\section{Klaseak JavaScript-en}
JavaScript ez da berez objektuetara zuzendutako programazio (OZP) lengoaia bat, baizik prototipoetan oinarritutakoa. Azken paradigma hori erabilita badago objektuak, herentzia, klaseak, etab. sortzea. Alegia, posible da objektuetara zuzendutako programazio-paradigmari jarraitzea, JavaScript-en helburua beste bat bada ere. Dena den, OZPa hain arrakastatsua bilakatu da JSn, non \index{ECMAScript}\index{ES2015} ECMAScript 2015\footnote{\href{https://262.ecma-international.org/6.0/}{https://262.ecma-international.org/6.0/}} bertsioan klaseak modu erraz batean programatzeko sintaxia onartu baitzen. Sintaxi berri horren bidez, klaseak, objektuak eta klaseen arteko herentzia inplementatzeko kodea erraztu egiten da erabat.

\subsection{Klaseen erazagupena}

Lauki izeneko klase bat erazagutzeko honako egiturari jarraituko diogu:

\index{class}

\begin{lstlisting}[language=JavaScript, numbers=none]
class Lauki {
  constructor(altuera, zabalera) {
    this.altuera = altuera;
    this.zabalera = zabalera;
  }
}
\end{lstlisting}

\index{constructor}\index{this}
Adibidean, Lauki klaseak eraikitzaile metodo bat dauka, \hlc[lightgray]{constructor} izenekoa (eraikitzailea beti deituko da \textit{constructor}). Barruan bi atributu deklaratu dira: altuera eta zabalera. Ohart zaitez atributuak erazagutzeko \hl{this} gako-hitza erabili dela, hots, \textit{this.altuera} eta \textit{this.zabalera}. Atributu hauen balioak esleitzeko, eraikitzaileari  parametro gisa pasatu dizkiogun aldagaien balioak erabili ditugu.

\index{new}
Berez, JSn, klase (\hl{class}) bat funtzio berezi bat besterik ez da. Funtzio horri \hl{new} gako-hitza erabiliz deitzen badiogu, funtzio horrek irudikatzen duen klasea instantziatuko dugu, objektua sortuz. Adibidez, Lauki klaseko laukia izeneko objektu bat instantziatzeko:

\begin{lstlisting}[language=JavaScript, numbers=none]
let laukia = new Lauki(3,4);
\end{lstlisting}

Funtzio arruntetan gertatzen den lez, klase-espresioak ere erabil daitezke:
\begin{lstlisting}[language=JavaScript, numbers=none]
let laukiKlasea = class Lauki {
  constructor(altuera, zabalera) {
    this.altuera = altuera;
    this.zabalera = zabalera;
  }
};

let laukia = new laukiKlasea(3,4);
\end{lstlisting}

\subsection{Get metodoak}
\index{get()}
\index{set()}

Hurrengo adibidean \textit{zabalera()} metodoa \hlc[lightgray]{get} hitzaz apaindu dugu. \hlc[lightgray]{get} gako-hitza da, sintaxia errazteko beste teknika bat. Horrela, \textit{laukia.zabaleraKalkulatu()} bezalako deia erabili ordez, zuzenean \textit{laukia.zabalera} erabili ahalko dugu. Gauza bera \hl{set} metodo bereziarekin. \hl{set} metodo bat erazagutuz, posible da balio bat esleitzea edozein atributuri modu laburrean (adib. \textit{laukia.altuera = 10;})

\begin{lstlisting}[language=JavaScript, numbers=none]
class Lauki {
  constructor(altuera, zabalera) {
    this.altuera = altuera;
    this.zabalera = zabalera;
  }

  set altuera(balioBerria) {
     this.altuera = balioBerria;
  }
  
  get zabalera() {
    return this.zabaleraKalkulatu();
  }

  zabaleraKalkulatu() {
    return this.altuera * this.zabalera;
  }

}
const laukia = new Lauki(5, 10);
console.log(laukia.zabalera); // (get metodoari deitu ondoren, emaitza=50)
laukia.altuera = 2; // set 
console.log(laukia.zabalera); // (get metodoari deitu ondoren, emaitza=20)

\end{lstlisting}

\subsection{Herentzia}

\index{herentzia}
\index{extends}
\index{ES2015}
Herentzia inplementatzeko sintaxia ere erraztu egin du ES2015 estandarrak, \hlc[lightgray]{extends} gako-hitzari esker.

\begin{lstlisting}[language=JavaScript, numbers=none]
class Karratu extends Lauki {
   zabaleraKalkulatu(){
       return this.altuera**2;
   }
   get alde() {
      return this.altuera;
   }
}

let karratu = new Karratu(10,10);
console.log(karratu.alde); // 10
console.log(karratu.zabalera); // 100

\end{lstlisting}

Ohart zaitez \textit{Karratu} klaseko \hl{zabaleraKalkulatu()} metodoan. Gainidatzitako metodo bat da, berez bere gurasoa (Lauki klasea) duen metodoa deuseztatu eta Laukiak bere metodo propioa inplementatzen du. Karratu klaseak \textit{set zabalera()} metodoa ere badu, Lauki klasetik heredatu egin duelako.

\section{Ariketak}

 Objektuetara zuzendutako programazioa erabili beharko duzu hainbat ariketa \newline JavaScript-en programatzeko.

\index{map()}
\index{filter()}
\index{reduce()}
\index{=>}
\index{gezi funtzioa}
\index{arrow}
\begin{enumerate}
\item ES6 sintaxia erabiliz, inplementatu Puntu izeneko klase bat, bi dimentsioko espazioan dagoen puntu bat irudikatzen duena. Puntu batek bi atributu ditu, \hl{x} eta \hl{y}, metodo eraikitzailetik pasatuko dizkiogunak. \textit{batu()} izeneko metodo bat ere badu, parametro gisa beste puntu bat hartzen duena eta emaitza gisa bi puntuen batura itzultzen duena. Alegia, beste puntu berri bat bueltatzen du, non x balioa bi puntuen x balioen batura den (eta gauza bera y puntuari dagokionez).

Beraz, kode hau exekutatzean: 

\begin{lstlisting}[language=JavaScript, numbers=none]
console.log(new Puntu(1, 2).batu(new Puntu(2, 1)))
\end{lstlisting}

Pantailan honako emaitza jaso behar dugu:
\begin{lstlisting}[language=JavaScript, numbers=none]
Puntu{x: 3, y: 3}
\end{lstlisting}

\item Gehitu \textit{hurrengoa()} izeneko metodo bat kontagailu objektuari. Metodo horrek \hl{kont} aldagaiaren uneko balioa itzuli behar du eta unitate batean handitu (erabili \hl{++} eragilea)
\begin{lstlisting}[language=JavaScript, numbers=none]
let kontagailu = {
 kont: 0
 [zure kodea hemen]
}
console.log(kontagailu.hurrengoa())
// -> 0
console.log(kontagailua.hurrengoa())
// -> 1
console.log(kontagailu.hurrengoa())
// -> 2
\end{lstlisting}

\item 
\index{programazio funtzionala}
map(), filter() eta reduce() programazio funtzionalean asko erabiltzen diren array klaseko hiru metodo dira. Webgune honetan \href{https://medium.com/poka-techblog/simplify-your-javascript-use-map-reduce-and-filter-bd02c593cc2d}{https://labur.eus/vGIW4} (medium.com) ikus ditzakezu metodo horien oinarrizko erabileraren adibideak. Horren inguruan zenbait ariketa proposatzen dira.

Honako objektu literala emanik (\hl{biltegia} izenekoa):
\begin{lstlisting}[language=JavaScript, numbers=none]

const biltegia = [
 {mota: "garbigailua", balioa: 5000},
 {mota: "garbigailua", balioa: 650},
 {mota: "edalontzia", balioa: 10},
 {mota: "armairua", balioa: 1200},
 {mota: "garbigailua", balioa: 77}
]

let guztiraGarbigailuenBalioa = zure kodea hemen;

console.log ( guztiraGarbigailuenBalioa ); // 5727 erantzuna espero da
\end{lstlisting}

Erabili \hl{filter()} eta \hl{reduce()} biltegiko garbigailuen balio totala lortzeko. 

\item Gezi funtzioa (\textit{arrow function})  \hl{=>} ES6 estandarrak ekarri duen eta oso preziatua  den funtzioa  bat da. Irakurri gehiago beraren inguruan hemen:  \newline \href{https://www.sitepoint.com/es6-arrow-functions-new-fat-concise-syntax-javascript/}{https://labur.eus/1XZv0} (sitepoint.com).

Jarraian, moldatu hurrengo kodea, \hl{arrow} funtzioa erabili dezan:

\begin{lstlisting}[language=JavaScript, numbers=none]
biltegia.forEach( function(item) {
console.log(item.balioa);
});
\end{lstlisting}

\item Egin hirugarren ariketaren errefaktorizazioa, \textit{arrow} funtzioa erabil dezan.


\end{enumerate}

