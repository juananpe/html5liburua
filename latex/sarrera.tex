\chapter*{Sarrera}
\addcontentsline{toc}{chapter}{Sarrera}  
%\section{Sarrera}

Liburu honen jatorria Bilboko Ingeniaritza Eskolan kokatzen da, duela hamar urte, webguneen garapenaren inguruko eskolak ematen hasi nintzenean. Bertan, HTML5 eta JavaScript APIak ikasteko apunteak eta ariketak prestatzen hasi nintzen, unibertsitateko eskoletan erabiltzeko eta etorkizunean, denbora lagun, liburu batean biltzeko asmoarekin. 

Esku artean duzu lan horren emaitza. Adibide praktikoz jositako liburu bat da, HTML5 lengoaia eta APIak ikas-irakasteko aproposa. Bertan, JavaScript-en inguruko ezagutzak freskatu ondoren (objektuetara zuzendutako programazioa, promesak, gertaerak, JSON formatua...), honakoak lantzen dira: Canvas (pantailan margotzeko), WebSocket-ak (sareko komunikazio azkarrak lortzeko, adibidez jokalari anitzeko web jokoak programatzeko), Audio eta Video APIak (multimedia-animazioak eta bideo-osagaiekin efektuak lortzeko), LocalStorage (nabigatzailean bertan informazioa gordetzeko), Geolocation APIa (uneko erabiltzailearen geokokapena lortzeko) eta beste hamaika APIren inguruko xehetasunak. Horiek guztiak \textit{front-end} edo bezeroaren aldeko programak egiteko erabiliko ditugu. Dena den, liburua idazten hasi nintzenean \textit{front-end} aldea programatzeko ezagutzak biltzea helburua bazen ere, berehala konturatu nintzen \textit{back-end} edo zerbitzariaren aldeko oinarrizko \textit{scriptak} programatzeko oinarrizko kontzeptuak ere landu beharko nituzkeela: bezeroan sortzen diren datuak jaso eta datu-base batean gordetzeko, erabiltzaileak kautotzeko, sareko konexioak kudeatzeko, etab. Hori dela eta, NodeJS —JavaScript zerbitzarian—, Express —web aplikazioak programatzen laguntzeko framework bat— eta MongoDB, NoSQL motako datu-base kudeatzaile baten inguruko oinarrizko kontzeptuak ere jorratuko dira.

Kapitulu bakoitzean, teoria apur bat landu ondoren, ariketa praktikoak proposatzen dira, gehienak soluzioarekin edo eskatzen den kodearen eskema orokorrarekin. Ariketetan lantzen diren kode-zatiak zure ordenagailuan edo online probatzea oso gomendagarria da. Kodea aztertuz, aldatuz, apurtuz, konponduz eta gauza berriak probatuz ikasten baita. 

\clearpage
\thispagestyle{empty}

Kodea probatzeko, hainbat tresna ditugu eskuragarri. Alde batetik, GitHuben argitaratu da ariketa eta adibideen kodea. Bestetik, zenbait kapitulutan agertzen diren aplikazioen kode-zatiak modu azkarrean probatu nahi izanez gero, ordenagailuan ezer instalatu gabe, codesandbox.io webgunean egitea proposatzen da.

Liburu honen bizitza ez da orri hauetan amaitzen. Web garatzaileen komunitateak ariketa berriak proposatuko ditu, HTML estandarraren bertsio berriak aterako dira, API eta funtzio berriekin, framework batzuk desagertu egingo dira, beste berri batzuei lekua egiteko... Hori guztia jaso eta dokumentatzeko webgune bat sortu da: \href{https://ikasten.io/html5/}{https://ikasten.io/html5/}.
Bertan argitaratuko dira ekarpen eta proiektu honen inguruko berri guztiak. Baita zuk bidalitakoak ere. Anima zaitez parte hartzera!\\

\begin{flushright}
Donostian, 2021eko azaroan
\end{flushright}
